\documentclass[]{article}
\usepackage{lmodern}
\usepackage{amssymb,amsmath}
\usepackage{ifxetex,ifluatex}
\usepackage{fixltx2e} % provides \textsubscript
\ifnum 0\ifxetex 1\fi\ifluatex 1\fi=0 % if pdftex
  \usepackage[T1]{fontenc}
  \usepackage[utf8]{inputenc}
\else % if luatex or xelatex
  \ifxetex
    \usepackage{mathspec}
  \else
    \usepackage{fontspec}
  \fi
  \defaultfontfeatures{Ligatures=TeX,Scale=MatchLowercase}
\fi
% use upquote if available, for straight quotes in verbatim environments
\IfFileExists{upquote.sty}{\usepackage{upquote}}{}
% use microtype if available
\IfFileExists{microtype.sty}{%
\usepackage{microtype}
\UseMicrotypeSet[protrusion]{basicmath} % disable protrusion for tt fonts
}{}
\usepackage[margin=1in]{geometry}
\usepackage{hyperref}
\hypersetup{unicode=true,
            pdfborder={0 0 0},
            breaklinks=true}
\urlstyle{same}  % don't use monospace font for urls
\usepackage{graphicx,grffile}
\makeatletter
\def\maxwidth{\ifdim\Gin@nat@width>\linewidth\linewidth\else\Gin@nat@width\fi}
\def\maxheight{\ifdim\Gin@nat@height>\textheight\textheight\else\Gin@nat@height\fi}
\makeatother
% Scale images if necessary, so that they will not overflow the page
% margins by default, and it is still possible to overwrite the defaults
% using explicit options in \includegraphics[width, height, ...]{}
\setkeys{Gin}{width=\maxwidth,height=\maxheight,keepaspectratio}
\IfFileExists{parskip.sty}{%
\usepackage{parskip}
}{% else
\setlength{\parindent}{0pt}
\setlength{\parskip}{6pt plus 2pt minus 1pt}
}
\setlength{\emergencystretch}{3em}  % prevent overfull lines
\providecommand{\tightlist}{%
  \setlength{\itemsep}{0pt}\setlength{\parskip}{0pt}}
\setcounter{secnumdepth}{0}
% Redefines (sub)paragraphs to behave more like sections
\ifx\paragraph\undefined\else
\let\oldparagraph\paragraph
\renewcommand{\paragraph}[1]{\oldparagraph{#1}\mbox{}}
\fi
\ifx\subparagraph\undefined\else
\let\oldsubparagraph\subparagraph
\renewcommand{\subparagraph}[1]{\oldsubparagraph{#1}\mbox{}}
\fi

%%% Use protect on footnotes to avoid problems with footnotes in titles
\let\rmarkdownfootnote\footnote%
\def\footnote{\protect\rmarkdownfootnote}

%%% Change title format to be more compact
\usepackage{titling}

% Create subtitle command for use in maketitle
\newcommand{\subtitle}[1]{
  \posttitle{
    \begin{center}\large#1\end{center}
    }
}

\setlength{\droptitle}{-2em}

  \title{}
    \pretitle{\vspace{\droptitle}}
  \posttitle{}
    \author{}
    \preauthor{}\postauthor{}
    \date{}
    \predate{}\postdate{}
  

\begin{document}

\%\textbackslash{}begin\{quote\}
\%\emph{Relations between variables are often more complex than simple bivariate relations.}
\%--- Fairchild and MacKinnon, 2009 \%\textbackslash{}end\{quote\}

\doublespacing

\section{Introduction}
\label{sec:Introduction}\subsection{Introduction}

\%\# The Problem The health and prevention sciences are increasingly
interested in not only \emph{if} one variable affects another but
\emph{how} that effect is transmitted. For example, research suggests
that chronic illness in adolescence can lead to poor health and
behavioral outcomes {[}@Pinquart2011{]}. But the question remains, how
does this affect take place? Does chronic illness decrease mental health
which in turn causes poor outcomes? Answering this question is not only
intellectually interesting but it can provide more meaningful avenues of
intervention. \emph{Mediation Analysis} is designed to help researchers
statistically investigate these questions.

Mediation analysis is a widely used technique that allows researchers to
investigate \emph{how} one variable may affect another through an
intermediate variable. Fields that seek malleable targets of
intervention {[}e.g., the prevention sciences; @Coie1993{]} find
mediation analysis to be an indispensable tool for it allows researchers
to evaluate the processes or pathways of an effect---of interventions,
risk-factors, and protective-factors alike {[}@Fairchild2009;
@Hayes2009; @MacKinnon2007; @Iacobucci2008book; @mackinnon2008intro;
@Shrout2002{]}. It uses predictors, mediators, and outcomes within a
single conceptual model where ``the independent {[}predictor{]} variable
influences the mediator, which in turn exerts an influence on the
dependent {[}outcome{]} variable,'' {[}@Serang2017, pg. 1{]}.

But, as it currently stands, mediation analysis is highly restricted to
be used with certain types of data in certain situations. Specifically,
when the hypothesized mediator and/or outcome is categorical or
otherwise non-normal (e.g., binary, count, multinomial) current
approaches are difficult to apply, are restricted to only a few useful
cases, and are even more difficult to interpret. If the goal of the
health and prevention sciences is to communicate important findings to
impact policy, intervention, behavior, and future research, both
\emph{utility} (the number of situations wherein it can be used) and
\emph{interpretability} (how easily the results can be understood and
applied) are key. Without these, results can be misunderstood and
misconstrued, leading to false beliefs and ineffective interventions.

Therefore, this project is designed to alleviate this issue through the
synthesis of two established methods, in effect creating a new framework
for mediation analysis that comfortably incorporates previous work.
Establishing this framework requires several key aims that this project
ultimately achieves: 1) the development of the general framework and its
software, 2) the evaluation of its performance in situations common to
health and prevention research, and 3) the application of it to
prevention data to demonstrate its use and to better understand
important relationships. Before establishing this new interpretable
framework, mediation analysis is discussed, highlighting its strengths
and current weaknesses.

\section{Mediation Analysis}\label{mediation-analysis}

Mediation analysis, as built on linear regression {[}@Hayes2009;
@Edwards2007{]}, combines two or more regression models to estimate the
full conceptual mediation model. It is sometimes referred to as
\emph{Conditional Process Analysis} (or Moderated Mediation, Mediated
Moderation) when combined with moderation {[}i.e., interaction effects;
@Hayes2013book{]} and is often portrayed via path diagrams and directed
acyclic graphs. Confirmatory analysis within mediation is well
established for a variety of situations {[}e.g., @Lockhart2011{]} while
exploratory analysis is beginning to take shape {[}@Serang2017{]}.
Confirmatory mediation has been applied often in health behavior
research---showing pathways leading to health-risk behavior such as drug
use {[}@Lockhart2017; @Luk2010; @Shih2010; @Wang2009{]}, tobacco use
{[}@Ennett2001{]}, and alcohol use {[}@Catanzaro2004{]}.

Knowing the pathway of effect allows clinicians, interventionists and
policymakers to target modifiable parts of the pathway. For example,
there is evidence that bully victimization in adolescence increases
depression, which subsequently increases drug use {[}@Luk2010{]}. In
this example, assuming no confounding, there are at least two immediate
targets of intervention: the victimization and the depression.
Interventions based on a model without the mediator will be incomplete
and may fail to alleviate the risk-factor(s). Further, without the
mediating effect included in the model, we are at risk of confounding,
causing our estimates to be misleading.

In its simplest form, as shown in Figure \ref{fig:simplemed}, X is the
predictor, M is the mediator (intermediate variable), and Y is the
outcome. The paths labeled \(a\) and \(b\) make up the mediated effect
(i.e., ``indirect'' effect) of X on Y whereas path \(c'\) is the direct
effect of X on Y {[}@Hayes2009{]}. The total effect is equal to
\(a \times b + c'\), which in linear models, should equal the \(c\)
estimate in the simple regression of \(Y = c_0 + cX + error\). It is
important to note that mediation analysis can become much more complex
than that in the figure, potentially for a more causal interpretation
{[}@Small2013; @Hayes2013book{]}.

\begin{figure}
  \centering
  \includegraphics[width=100mm]{figures/Fig_SimpleMediation.jpg}
  \caption{Path diagram of a simple mediation analysis model with a single predictor, a single mediator, and a single outcome.}
  \label{fig:simplemed}
\end{figure}

\subsection{Definitions}\label{definitions}

Before discussing mediation further, it is helpful to note some
terminology that are often used in the field. In Figure
\ref{fig:simplemed}, X is a predictor or an exogenous variable (i.e., a
variable that is not predicted or influenced by something in the model;
``independent variable'') while M and Y are mediators and outcomes,
respectively. These are also known as endogenous variables (i.e.,
variables that are, in part, predicted or influenced by other variables
in the model). These distinctions are useful as the frameworks and
assumptions of the models are discussed.

\section{Frameworks}\label{frameworks}

Two highly related frameworks exist to perform mediation analysis
{[}@Iacobucci2008book{]}. First, as mentioned previously, mediation
analysis can be built on linear regression including ordinary least
squares (OLS) and generalized linear modeling {[}GLM; @Hayes2009;
@Hayes2013book{]}. This requires separate models for the \(a\) paths and
the \(b\) and \(c'\) paths, fit independently, to be combined into one
mediation model. This approach is flexible in terms of the types of
variables and model specifications as compared to the other---structural
equation modeling (SEM). For example, performing moderated mediation is
more straightforward in this framework {[}@Hayes2013book;
@Edwards2007{]} than in SEM. Ultimately, the regression-based framework
is what this project builds upon.

Under the SEM paradigm, all the paths are simultaneously estimated,
sometimes providing more statistical power
{[}@Iacobucci2008book{]}.\footnote{The idea that SEM is "superior" to the regression paradigm was refuted by Hayes (2013) by noting that in most situations differences in the estimation is extremely minor and will not alter the conclusions. This can be seen in the small effect sizes presented in Iacobucci et al. (2007). Further, additional assumptions inherent in the SEM approach may not hold, although some are not easily tested (e.g., multivariate normality). With this said, SEM still provides a powerful framework for mediation analysis.}
This approach notably allows more testing of the full model fit and can
easily include latent variables but assumes, in general, that all
variables are continuous with a multivariate normal distribution. This
is a strict assumption that is difficult to assess. However, it has
extensions allowing for categorical (generally ordinal) variables to be
included, although this changes the estimation procedure. The issues
relating to categorical mediators/outcomes are discussed in the
``Analytic and Interpretation Issues with Mediation Analysis'' section.

\section{Assumptions}\label{assumptions}

In his 2008 book ``Introduction to Mediation Analysis,''
@mackinnon2008intro discusses the
assumptions\footnote{The assumptions described herein are for both the regression and SEM frameworks for mediation, although, as noted above, SEM has a few additional assumptions as well.}
of the mediation modeling procedure. Of these primary assumptions, note
that there are no major differences from the assumptions of regression
analysis.

\begin{enumerate}
\def\labelenumi{\arabic{enumi}.}
\tightlist
\item
  \emph{Correct Functional Form}. In general, mediation assumes a linear
  relationship between predictors and mediators/outcomes. This can be
  adjusted using transformations or, more pertinently, generalized
  linear models (e.g., logistic regression). @mackinnon2008intro also
  points out that it is assumed the relationships are additive; if they
  are not, then the correct interactions (moderators) need to be
  included in the model specification. This, in many ways, needs to be
  driven by theory and prior literature {[}@Lockhart2011{]}.
\item
  \emph{No Omitted Influences}. A key to any mediation analysis is that
  variables that: 1) correlate with both the predictor and the mediator
  (path \(a\)), 2) correlate with both the mediator and the outcome
  (path \(b\)), or 3) correlate with both the predictor and the outcome
  (path \(c'\)) are included in the model. A more general form of this
  assumption has been termed ``sequential ignorability''
  {[}@Imai2010a{]}. This more general form includes a sensitivity
  analysis to assess how important deviations from this assumption are
  on the conclusions {[}@Imai2010a; @Imai2010b{]}.
\item
  \emph{Accurate Measurement}. Random measurement error produces
  attenuated paths (in large sample sizes) and random bias (in small
  sample sizes) in regression {[}@Loken2017{]} and therefore can affect
  the paths in various ways (e.g., attenuate the \(b\) path which can
  inflate the \(c'\) path). When possible, reliable measures and/or
  proper latent variable modeling should be used for this assumption to
  be met.
\item
  \emph{Well-Behaved Residuals}. The residuals are assumed to be random,
  ``have constant variance at each value of the predictor variable'' and
  ``residual error terms are uncorrelated across equations'' (pg. 55).
  The assumption about uncorrelated errors can stem from ``No Omitted
  Influences'' for, if there are omitted variables in both equations,
  the error terms will correlate. This is one of the few assumptions
  that can be investigated in many situations.
\end{enumerate}

With the addition of \emph{temporal precedence} (predictor comes before
mediator) and \emph{appropriate measurement timing} (the mediator is
measured at the appropriate time when the effect of the predictor has
occurred), the resulting estimates are asymptotically (i.e., with a
large enough sample size) unbiased, allowing proper (causal) inference
regarding the effects' magnitude and direction
{[}@mackinnon2008intro{]}. This interpretation, to aid in
reproducibility, needs to be highlighted with the associated uncertainty
(e.g., confidence intervals) in a meaningful metric.

\subsection{Other Considerations}\label{other-considerations}

\subsubsection{Causality}\label{causality}

To discuss causality in mediation analysis, one should be familiar with
the \emph{counter-factual} framework {[}or sometimes referred to as the
potential outcomes framework; @Hofler2005{]}. As @Imai2010a states:
``the causal effect \ldots{} can be defined as the difference between
two potential outcomes: one that would be realized {[}in the
intervention{]} and the other that would be realized if {[}not in the
intervention{]},'' (pg. 3). In other words, the causal effect is the
difference in potential outcomes depending on the predictor (e.g., an
intervention). In reality, only one such outcome is observed---if
individual ``i'' is assigned to the treatment group, we only observe the
outcome from the treatment group and not from the control group.
@Imai2010a continues: ``Given this setup, the causal effect of the
{[}intervention{]} can be defined as \(Y_i(1) - Y_i(0)\). Because only
either \(Y_i(1)\) or \(Y_i(0)\) is observable \ldots{} researchers often
focus on the identification and estimation of the average causal
effect.'' If the conditions are randomly assigned, this is simply
\(E(Y_i(1) - Y_i(0))\), or the expected value across multiple
individuals and/or observations.

The counter-factual framework helps clarify causality in mediation
analysis by defining the necessary conditions. Using this framework,
@Imai2010b demonstrated that \emph{sequential ignorability} (essentially
the assumption that there are no omitted influences) is required for a
causal interpretation in mediation analysis. However, this assumption is
difficult to assess. Because of this difficulty, @Imai2010a developed a
general mediation model that allows a researcher to assess how
deviations from it, via sensitivity analysis, affect the estimates.

As will be shown, the present project incorporates the counter-factual
framework intuitively. Because of its importance, this will be discussed
more in the next chapter.

\subsubsection{Modeling}\label{modeling}

@Shrout2002 highlight a number of other important considerations in
mediation analysis. First, multi-collinearity can produce problems,
especially when it occurs between predictors and mediators. It can
distort the statistical power of the analysis, potentially producing
misleading results. The second consideration is suppression:
``Suppression occurs when the indirect effect \(a \times b\) has the
opposite sign of the direct effect,'' (pg. 430). This can, if not
interpreted correctly, produce confusing estimates (e.g., a positive
indirect path and a negative direct path).

@Shrout2002 also recommend using bootstrapping {[}also @Hayes2009;
@Hayes2013book{]} to understand the variability in the estimates. This
is due to the asymmetric distribution of indirect effects {[}see Figure
6 in @Shrout2002{]} that boostrapping can handle naturally.
Bootstrapping uses repeated random sampling of the data with replacement
and estimates the model on that sampled data. Generally, between 500 and
10,000 bootstrapped samples are used to get an accurate confidence
interval. In regards to mediation analysis, bootstrapping produces as
accurate (or more accurate) Type-I error rates than other methods in
mediation analysis. Because of this, bootstrapping plays a major role in
this project.

Two other considerations should be made regarding mediation analysis.
First, only the predictors can be randomized (i.e., the mediator cannot
be randomized in most situations). That is, even when the \(a\) and
\(c'\) paths can portray an experimental manipulation, the \(b\) path(s)
cannot. Therefore, the need for proper covariates, interpretation,
reporting, and replication is even more important in mediation analysis.
David MacKinnon said it well: ``It is not likely that a true mechanism
can be demonstrated in one statistical analysis. \ldots{} These analyses
inform the next experiment that provides more information,''
{[}@mackinnon2008intro, pg. 67{]}. Therefore, the conceptual model must
be considered carefully in light of theory, prior literature, and proper
covariates {[}@Lockhart2011{]}. @Iacobucci2008book also recommends to
evaluate competing models and theories---thus presenting the effects and
paths in light of alternative model specifications.

Even when done properly, replication of mediated effects is important
{[}@mackinnon2008intro{]}. To help make the replications most useful,
the interpretation---comprising the magnitude and direction of the
effect---needs to be reported with the proper uncertainty and include
information on a) bivariate correlations, b) information on all relevant
paths (even non-significant ones), c) must include information on the
process of variable and covariate selection, d) report standardized and
unstandardized results, and e) provide de-identified data and code {[}if
possible{]}. It is important to note that much of this information can
be included as supplemental material. In this way, results are reported
that can be combined with others in order to provide ``convincing
evidence of {[}or lack of{]} a mediating mechanism'',
{[}@mackinnon2008intro, pg. 67{]}.

Second, interpretation is built on combining multiple estimates, and
often subsequently comparing those combinations with other estimates.
For example, the indirect effect is a combination of the \(a\) and \(b\)
paths and is often compared with the \(c'\) path. Therefore, if either
of the \(a\) or \(b\) paths are in units that cannot be easily combined,
the interpretation quickly becomes very difficult (see ``Analytic and
Interpretation Issues with Mediation Analysis'' section). This is
particularly true when the mediator(s) and/or outcome is non-normal.

This second hurdle, that of interpretation, is particularly important in
this project. Below, the general interpretation guidelines are
discussed, followed by when these guidelines are not straightforward.

\section{Interpretation}\label{interpretation}

In linear models, the interpretation is simple, straightforward and
intuitive. The \(a\) path coefficient means: ``for a one unit change in
X there is an associated \(a\) units change in the mediator.'' Likewise,
the \(b\) path coefficient means: ``for a one unit change in M there is
an associated \(b\) units change in the outcome, controlling for the
effect of X.'' Finally, the \(c'\) path is: ``for a one unit change in
X, controlling for the effect of M, there is an associated change of
\(c'\) units in the outcome.'' The indirect effect is \(a \times b\);
the total effect is \(a \times b + c'\).

Each element of the mediation (i.e., the indirect, the direct, and total
effects and also the individual \(a\), \(b\), and \(c'\) paths) needs to
be considered without only trying to answer: ``Is there a mediated
effect?'' Otherwise, researchers can lose sight of the complete story.
For example, the various \(a\) paths may be important on their own
(e.g., if the \(a\) path effect size is small then maybe the predictor
is not a beneficial place to focus an intervention even though the
effect is significant). Therefore, understanding a mediated effect is
best told through several avenues: the indirect, direct, and total
effects; the individual paths; these effects and paths in light of
covariates; among others. This approach is also best if those effects
are in meaningful and intuitive metrics.

However, once the analysis ventures into non-normal, non-linear
relationships, the interpretation becomes more difficult---particularly
when it comes to the indirect and total effects. For example, if the
mediator is binary, often logistic regression is used to assess the
\(a\) path. But that changes the \(a\) path interpretation to: ``for a
one unit change in X, there is an associated \(a\) log odds units change
in the mediator.'' This interpretation is anything but intuitive. In
general, the log odds are transformed into odds ratios, which improve
the interpretation. But these units do not mix well with other units.
This is detrimental to understanding the indirect and total effects as
will be discussed further in the sections below.

\section{Analytic and Interpretation Issues with Mediation
Analysis}\label{analytic-and-interpretation-issues-with-mediation-analysis}

\subsection{Categorical and Non-Normal
Mediators/Outcomes}\label{categorical-and-non-normal-mediatorsoutcomes}

Mediation analysis is more difficult when the mediator and/or the
outcome is not continuous, including binary variables (e.g., an
individual either uses marijuana or not), ordinal variables (e.g., the
self-reported confidence in social settings), other polytomous variables
(e.g., sub-types of a disease), and count variables (e.g., number of
hospital visits). These data situations are difficult because mediation
analysis requires the mediators and/or outcomes to be continuous and
approximately normal (to meet the assumption of well-behaved residuals).

There are several strategies taken in the literature to address this
problem. However, each makes its own set of assumptions and each
contains limitations in interpretation. The variability in approaches
and the subsequent interpretations make combining results across studies
far more difficult---reducing the chance to concretely show
relationships via meta-analyses and systematic reviews. The difficulty
of these data situations are likely reducing reliability and
interpretability for a number of reasons:

\begin{itemize}
\tightlist
\item
  It may be easier to ignore the assumptions that are violated when
  using categorical mediators and/or outcomes. Results from these
  analyses may not be valid.
\item
  Different approaches produce varying assumptions and interpretations.
  This can be difficult for other researchers, clinicians, lawmakers,
  and laypersons to keep in mind, possibly leading to misunderstandings
  regarding results and their validity.
\item
  Analyses with categorical outcomes are not typically well-emphasized
  in graduate training, even in more simple modeling techniques, not to
  mention more complex techniques like mediation analysis. With fewer
  individuals well-trained, more errors are likely in analyzing these
  data.
\item
  The interpretation regarding analyses with categorical outcomes is
  often far less intuitive than with continuous outcomes. This can
  produce a higher cognitive load for both the researchers and those
  utilizing the study.
\end{itemize}

With this in mind, the following subsection discusses the current
approaches to mediation analysis with categorical mediators and/or
outcomes and the assumptions these approaches make.

\subsection{Current Approaches}\label{current-approaches}

\begin{quote}
``The quest for sound methods of incorporating categorical variables is
perhaps the last dilemma in mediation analysis that lacks a strong
solution---it's the `final frontier,'\,'' {[}@Iacobucci2012, pg. 583{]}.
\end{quote}

Although there are possibly other ``frontiers'' in mediation analysis,
categorical mediators and/or outcomes certainly produce several
challenges. @Iacobucci2008book {[}-@Iacobucci2012{]} thoroughly
discusses the issues of assessing categorical variables within a
mediation analysis and some of the current practices along with their
associated
problems\footnote{For the present, only situations when a mediator or outcome is categorical is under consideration since a categorical predictor poses very few problems (Iacobucci, 2012).}.
Four approaches are of note here: 1) a series of logistic regression
models, 2) polychoric correlation in SEM, 3) the method suggested by
@Iacobucci2012 regarding standardization, and 4) interpreting each path
separately. A fifth, but certainly least, is to ignore the distribution
of the outcomes and purposefully misspecify the model. These are
highlighted in Table \ref{tab_approaches}.

\begin{table}[tb]
\centering
\caption{The various approaches to handling mediation with categorical mediators/outcomes.} 
\label{tab_approaches}
\begin{tabular}{p{40mm}p{45mm}p{55mm}}
\toprule
Approach & Pros & Cons \\ 
\midrule
1. Series of logistic regressions & Simple to apply in most software & Cannot obtain indirect effect size, only useful in few situations \\ 
2. SEM's approach (polychoric correlation) & Powerful, well-designed, Easy to implement with proper software & Only works with ordinal variables, only standardized effect sizes \\ 
3. Standardize the coefficients & Provides significance test of indirect effect & Assumptions regarding distributions, difficult to interpret beyond p-value \\ 
4. Interpret each path separately & Simplest approach with proper models & Ignores some information, cannot obtain indirect effect size \\ 
5. Pretend all variables are continuous & Simplest approach & Purposeful mis-specification, poor model fit \\ 
\bottomrule
\end{tabular}
\end{table}

Of these approaches, only the SEM approach allows any proper estimation
of effect sizes. This approach was developed by @Muthen1984, which uses
polychoric correlations (for ordered categorical variables) within the
structural equation modeling framework. In this approach, @Muthen1984
assumes that the ordered categorical manifest variable results from a
continuous latent variable, ``with observed categorical data arising
through a threshold step function,'' {[}@Iacobucci2012, pg. 583{]}. In
many cases, a continuous latent variable is a reasonable assumption. For
example, in a binary variable describing whether an adolescent has ever
smoked marijuana, assuming a continuous latent variable comprised of the
probability of smoking may adequately represent reality. This method
relies on using a probit regression ``to model the relationship between
the observed categorical variable and the latent \emph{normally
distributed} variable,'' {[}@mackinnon2008intro, pg. 320, emphasis
added{]} and requires large sample sizes {[}@Iacobucci2008book{]}. Using
this latent-observed sub-model, the probit is used as the threshold
value to estimate the full model. Notably, both the assumption of a
normally distributed latent variable and the requirement of large sample
sizes are limitations of this approach.

The other approaches, including the series of logistic regression models
and the interpretation of each path separately, do not allow any effect
size estimation of indirect effects. Although there are some ways to
discuss the proportion of mediation {[}@Ford2012{]}, this still does not
provide any measures of effect size. The third approach, recommended by
@Iacobucci2012, proposed using a new standardization solution using
logistic regression output (for a categorical variable) that can combine
linear and logistic models' estimates. It is built on the same idea as
the Sobel test {[}@Sobel1982; @mackinnon2008intro{]} but is more
flexible; for ``the mechanics of testing for mediation do not {[}need
to{]} change whether the variables are continuous or categorical or some
mix,'' {[}@Iacobucci2012, pg. 593{]}. However, @MacKinnon2012 criticized
this approach. Ultimately, the most fatal flaw may be that a focus on
the significance (relying on Null Hypothesis Significance Testing) of
the effects without regard to their effect size in meaningful terms is
less interpretable and impactful.

In the end, none of these approaches can flexibly handle categorical
variables of various kinds (e.g., binary, multinomial) or other
non-normal distributions (e.g., costs, counts); none can produce
intuitive and meaningful effect sizes and confidence intervals across
these variable types; and none can consistently combine two differing
types of estimates (e.g., binary mediator with continuous outcome, count
mediator with binary outcome). Although the current approaches are
useful in some situations---particularly the SEM approach---a more
complete framework is needed.

\section{Conclusions}\label{conclusions}

Mediation analysis is a powerful framework for understanding the
processes by which one variable influences another. The assumptions are
not much more than that of regression analysis. The interpretation, in
linear models, is straightforward and simple. However, once the analysis
ventures into non-normal, non-linear relationships, the interpretation
becomes more difficult---particularly when it comes to the indirect and
total effects.

In the end, @Iacobucci2012 is correct in saying this problem ``lacks a
strong solution'' (pg. 583). Although important information can be
obtained from the current methods, mediation analysis with categorical
mediator(s) and/or outcome(s) still misses the mark on intuitive,
meaningful effect sizes.

This project aims to alleviate these issues by integrating a
post-estimation approach known as \emph{Average Marginal Effects} (AMEs)
within mediation analysis. This integration can allow simple and
meaningful interpretation across variable types and combinations thus
far shown to be problematic. The following chapter introduces AMEs,
showing their benefit in interpretation and reporting when working with
non-normal variables within generalized linear models.

\singlespacing


\end{document}
